\documentclass{article}
\usepackage{fullpage}

\newcommand{\header}[2]{%
    \begin{flushright}
    #1 \\
    #2 \\
    \today \\
    \end{flushright}%
}

\newcommand{\xargs}{\texttt{xargs} }

\begin{document}
\header{Josh Kunz}{xargs Analysis}
%
\vspace{-2ex}
\section{Overview}
\vspace{-1ex}
\xargs is a standard UNIX utility (from the findutils utility package) for
converting file contents into arguments for other programs. Typically \xargs is
employed when the user wishes to run a command over a set of input files. For
example, if a user had a directory full of \texttt{*.wav} files that they wanted
to convert to \texttt{*.mp3} files, they might use xargs like so:

\begin{center}
\verb$find . -name "*.wav" -print0 | xargs -0 -n1 lame -V0$
\end{center}

The \texttt{find} utility is used to build a list of \texttt{wav} files which
is then passed to \texttt{xargs}. \xargs converts this input to a list of arguments
(in this case each files is an argument) and then passes those arguments to 
utility program, in this case a popular wav to mp3 conversion tool \texttt{lame}.
Here we use \texttt{xarg}'s \verb|-n| argument to ensure that only one file is
passed to \texttt{lame} at a time. Without this option \xargs will try to pass
as many arguments to the utility program as it can.

From this command we can already see some peculiarities of the \xargs command,
for example \texttt{xargs}'s \verb|-0| option. Under normal operation, input to 
\xargs is \textit{parsed} according to the conventions normally used by shells.
Input lines are split into arguments on whitespace (so the line \texttt{abc def}
would be parsed a two inputs \texttt{abc} and \texttt{def}), whitespace can
be escaped with quotes (\texttt{"abc def"} becomes one argument) and quote
characters can be escaped with backslashes (\texttt{"abc \textbackslash" def"}
is parsed as \texttt{abc " def}).

What this means, is that if you want run a command over a set of files, those
files have to be properly escaped before they are given as input to \xargs. This
can be very confusing for users of \xargs who might expect it to simply use each
line as argument instead of parsing each line into arguments. This confusing
behavior should be considered before \xargs is relied upon.

As can be seen above \xargs has added a work around for this problem, the 
\verb|-0| option. This option causes \xargs to treat everything between two 
null-terminators (or a null-terminator and an \texttt{EOF}) as an argument, no
parsing is done on the input.

\vspace{-2ex}
\section{Analysis}
\vspace{-1ex}
The lab currently has two versions of \xargs installed 4.1, the oldest
release still available from the findutils project. It was released in 1994. The
lab \textit{also} has version 4.4.2 installed the newest version of the \xargs
utility. However, the default version is version 4.1. Since this version is
severely outdated, and a newer version has already been installed, version 4.4.2
should be configured as the default as soon as possible. I was not able to find
any bugs that directly affected security, but this very outdated version does
have a number of logic bugs in it.

\begin{center}
\parbox[c][1cm][c]{0.7\linewidth}{
    {\small \it Note: version 4.1 was in-fact so old that I was unable to compile it
     on any relatively `modern' system I control. The analysis presented below
     was done on version 4.2.25, the next release version.}
}
\end{center}

\xargs itself already comes with a fairly comprehensive set of tests built
on the DejaGnu test system. I compiled \xargs with the \texttt{clang} compiler's 
undefined behavior sanitizer, and ran the test-suite. To my surprise, 
\xargs didn't invoke \textit{any} undefined behavior. I also compiled \xargs 
using \texttt{gcc}'s code coverage tools, and then ran the test-suite. 
\texttt{xargs}'s supplied test-suite achieved 73\% code coverage. Most 
un-covered code was system-call error handling code, and code whose reachability
is platform dependent (though most of that code was quite simple). I attempted
to test system-call error handling with the Murphy fuzzing tool, but was unable
to get the tool working within the alloted time.

I also tested \xargs for memory related bugs by running it under valgrind. I supplied
a several gigabyte input file and detected no memory leaks. In fact, I noticed 
that \xargs allocates the same amount of memory for every run, regardless of 
input size. From inspection of the code, it looks like this is because \xargs
enforces hard limits on the size (and number) of arguments passed to its child
program.

One last note, \xargs has built in support for concurrent execution. If the \verb|-P|
flag is passed to the application, \xargs will run multiple instances of the 
utility application. \xargs uses the traditional UNIX fork/exec approach to 
concurrency so the main vector for race conditions will be through the filesystem.
If two applications access the same file, race conditions may be triggered, however
this is a problem inherent in concurrently running arbitrary utility programs.
If synchronization is needed, a custom tool should probably be used instead.

From my cursory inspection of the \xargs utility, it appears to be a well
tested, and robust (though not stylistically well-written) piece of code.
It's unlikely that \xargs cause any issues, yet some
errors may results from the confusing way in which \xargs handles input.

\end{document}
