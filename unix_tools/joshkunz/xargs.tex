\documentclass{article}
\usepackage{fullpage}

\newcommand{\header}[2]{%
    \begin{flushright}
    #1 \\
    #2 \\
    \today \\
    \end{flushright}%
}

\newcommand{\xargs}{\texttt{xargs} }

\begin{document}
\header{Josh Kunz}{xargs Analysis}

\section{Overview}
\xargs is a standard UNIX utility (from the findutils utility package) for
converting file contents into arguments for other programs. Typically \xargs is
employed when the user wishes to run a command over a set of input files. For
example, if a user had a directory full of \texttt{*.wav} files that they wanted
to convert to \texttt{*.mp3} files, they might use xargs like so:

\begin{center}
\verb$find . -name "*.wav" -print0 | xargs -0 -n1 lame -V0$
\end{center}

The \texttt{find} utility is used to build a list of \texttt{wav} files which
is then passed to \texttt{xargs}. \xargs converts this input to a list of arguments
(in this case each files is an argument) and then passes those arguments to 
utility program, in this case a popular wav to mp3 conversion tool \texttt{lame}.
Here we use \texttt{xarg}'s \verb|-n| argument to ensure that only one file is
passed to \texttt{lame} at a time. Without this option \xargs will try to pass
as many arguments to the utility program as it can.

From this command we can already see some peculiarities of the \xargs command,
for example \texttt{xargs}'s \verb|-0| option. Under normal operation, input to 
\xargs is \textit{parsed} according to the conventions normally used by shells.
Input lines are split into arguments on whitespace (so the line \texttt{abc def}
would be parsed a two inputs \texttt{abc} and \texttt{def}), whitespace can
be escaped with quotes (\texttt{"abc def"} becomes one argument) and quote
characters can be escaped with backslashes (\texttt{"abc \textbackslash" def"}
is parsed as \texttt{abc " def}).

What this means, is that if you want run a command over a set of files, those
files have to be properly escaped before they are given as input to \xargs. This
can be very confusing for users of \xargs who might expect it to simply use each
line as argument instead of parsing each line into arguments. This confusing
behaviour should be considered before \xargs is relied upon.

As can be seen above \xargs has added a work around for this problem, the 
\verb|-0| option. This option causes \xargs to treat everything between two 
null-terminators (or a null-terminator and an \texttt{EOF}) as an argument, no
parsing is done on the input.

\section{Analysis}
The lab currently has two versions of \xargs installed 4.1, the oldest
release still available from the findutils project. It was released in 1994. The
lab \textit{also} has version 4.4.2 installed the newest version of the \xargs
utility. However, the default version is version 4.1. Since this version is
severely outdated, and a newer version has already been installed, version 4.4.2
should be configured as the default as soon as possible. I was not able to find
any bugs that directly affected security, but this very outdated version does
have a number of logic bugs in it.

\begin{center}
\parbox[c][1cm][c]{0.8\linewidth}{
    {\it Note: version 4.1 was in-fact so old that I was unable to compile it
     on any relatively 'modern' system I control. The analysis presented below
     was done on version 4.2.25, the next release version.}
}
\end{center}

\end{document}
